\section{Годограф Найквиста}

В соответствии с вариантом придумаем три объекта пятого порядка с 
тремя вещественными полюсами и двумя комплексно-сопряженными. Все следующие
передаточные функции (ПФ) объектов были подобраны вручную.
 
\subsection{Объект 1}

ПФ с четырьмя неустойчивыми полюсами у разомкнутой системы и один неустойчивый полюс у замкнутой,
разомкнутая система:
\begin{equation*}
    W_1(s)=\frac{100s^4+100s^3+100s^2+100s-100}{(s-1-j)(s-1+j)(s-2)(s-3)(s+1)}
    = \frac{100s^4+100s^3+100s^2+100s-100}{s^5-6s^4+11s^3-4s^2-10s+12}.
\end{equation*}

\subsection{Объект 2}

ПФ без неустойчивых полюсов у разомкнутой системы и с одним неустойчивым полюсом у замкнутой,
разомкнутая система:
\begin{equation*}
    W_2(s)=\frac{100s^4-100s^3-10000s^2-1000s-100}{(s+1+j)(s+1-j)(s+2)(s+3)(s+4)}=
    \frac{100s^4-100s^3-10000s^2-1000s-100}{s^5+11s^4+46s^3+94s^2+100s+48}.
\end{equation*}

\subsection{Объект 3}

ПФ с четырьмя неустойчивыми полюсами у разомкнутой системы и без таковых у замкнутой,
разомкнутая система:
\begin{equation*}
    W_3(s)=\frac{100s^4-100s^3-10000s^2-1000s-100}{(s-1-j)(s-1+j)(s-2)(s-3)(s+1)}
    = \frac{100s^4-100s^3-10000s^2-1000s-100}{s^5-6s^4+11s^3-4s^2-10s+12}.
\end{equation*}
