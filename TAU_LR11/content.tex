\section{Компенсирующий регулятор по состоянию}
\subsection{Анализ системы}
Рассмотрим систему 
\begin{equation}
    \dot x=Ax+Bu+B_fw_f,\quad x(0)=\begin{bmatrix}
        0&0&0
    \end{bmatrix}^T,
    \label{eq:sys1}
\end{equation}
генератор внешнего возмущения
\begin{equation*}
    \dot w_f=\Gamma w_f,\quad w_f(0)=\begin{bmatrix}
        1&1&1&1
    \end{bmatrix}^T
\end{equation*}
и виртуальный выход вида
\begin{equation*}
    z=C_Zx,
\end{equation*}
где
\begin{equation*}
    A=\begin{bmatrix}
        3&5&4\\
        -2&-4&-5\\
        2&2&3
    \end{bmatrix},\quad
    B=\begin{bmatrix}
        2\\-1\\1
    \end{bmatrix},\quad
    B_f=\begin{bmatrix}
        -2&0&0&2\\
        -2&0&0&0\\
        0&0&0&0
    \end{bmatrix},\quad
\end{equation*}
\begin{equation*}
    \Gamma=\begin{bmatrix}
        35&56&22&-42\\
        -11&-17&-7&12\\
        -6&-10&-5&10\\
        11&18&6&-13
    \end{bmatrix},\quad
    C_Z=\begin{bmatrix}
        2&3&3
    \end{bmatrix}.
\end{equation*}
Собственные числа матрицы $\Gamma$
\begin{equation*}
    \sigma(\Gamma)=\{\pm3i,\ \pm i\},
\end{equation*}
внешнее возмущение имеет вид суммы гармоник с увеличевающейся
со временем амплитудой:
\begin{equation*}
    w(t)=t(a\cos(t)+b\sin(t) + c\cos(3t)+d\sin(3t)).
\end{equation*}
Построим схему моделирования системы \eqref{eq:sys1},
замкнутой компенсирующим регулятором
\begin{equation}
    u=K_1x+K_2w_f,
    \label{eq:reg1}
\end{equation}
обеспечивающим выполнение целевого условия
\begin{equation*}
    \lim_{t\rightarrow\infty}z(t)=0.
\end{equation*}
Схему можно увидеть на \autoref{fig:sys1}.
Проверим стабилизируема ли система (пара ($A$, $B$)).
\begin{equation*}
    \sigma(A)=\{2\pm i,\ -i\},
\end{equation*}
\begin{equation*}
    H_1=\begin{bmatrix}
        -1 + i & -5 & -4 & 2 \\
        2 & 6 + i & 5 & -1 \\
        -2 & -2 & -1 + i & 1
    \end{bmatrix},\quad \text{rank}(H_1)=3,
\end{equation*}
\begin{equation*}
    H_2=\begin{bmatrix}
        -1 - i & -5 & -4 & 2 \\
        2 & 6 - i & 5 & -1 \\
        -2 & -2 & -1 - i & 1
    \end{bmatrix},\quad \text{rank}(H_2)=3,
\end{equation*}
\begin{equation*}
    H_3=\begin{bmatrix}
        -5 & -5 & -4 & 2 \\
        2 & 2 & 5 & -1 \\
        -2 & -2 & -5 & 1
    \end{bmatrix},\quad \text{rank}(H_3)=2.
\end{equation*}
Пара ($A$, $B$) не является полностью управляемой, но
стабилизируема, этого достаточно для синтеза регулятора.

\begin{figure}[H]
    \centering
    
    \caption{Схема моделирования системы \eqref{eq:sys1},
    замкнутой компенсирующим регулятором \eqref{eq:reg1}}
    \label{fig:sys1}
\end{figure}

\subsection{Синтез ``feedback''-компоненты}

Синтезируем «feedback»-компоненту $K_1$ компенсирующего регулятора \eqref{eq:reg1}
с помощью уравнения Сильвестра:
\begin{equation*}
    AP-P\Gamma_R=BY,\quad K=-YP^{-1}.
\end{equation*}
Но сначала нужно усечть систему. Найдем жордановые формы матриц:
\begin{equation*}
    A_J=\begin{bmatrix}
        -2&     0&     0\\
        0&     2&    -1\\
        0&     1&     2   
    \end{bmatrix},\quad
    B_J=\begin{bmatrix}
        0\\
        0.7071\\
       -2.1213
    \end{bmatrix},\quad
    P_J = \begin{bmatrix}
        -1&    0.7071&   -0.7071\\
        1&   -1.4142&         0\\
             0    &1.4142&         0
    \end{bmatrix},
\end{equation*}
где $P_J$ - матрица перехода. Усечем до следущих матриц:
\begin{equation*}
    A_j=\begin{bmatrix}
        2&    -1\\
        1&     2   
    \end{bmatrix},\quad
    B_j=\begin{bmatrix}
        0.7071\\
       -2.1213
    \end{bmatrix}.
\end{equation*}
возьмем следующие матрицы $\Gamma_R$ и $Y$, чтобы итоговый спект был устойчив и
пара ($\Gamma_R$, $Y$) наблюдаема
\begin{equation*}
    \Gamma_R=\begin{bmatrix}
        -10 & 1\\0&-10
    \end{bmatrix},\quad
    Y=\begin{bmatrix}
        1 & 0
    \end{bmatrix}.
\end{equation*}
Используя CVX получим следующую матрицу регулятора
\begin{equation*}
    K_{1_j}=\begin{bmatrix}
        -64.0645&-10.0410
    \end{bmatrix},
\end{equation*}
расширим и вернем в изначальный базис
\begin{equation*}
    K_1=\begin{bmatrix}
        0&K_{1_j}
    \end{bmatrix}\times P_J^{-1}=
    \begin{bmatrix}
        14.2001&	14.2001&	-38.2004
    \end{bmatrix}.
\end{equation*}
Проверим, что получили желаемый спектр
\begin{equation*}
    \sigma(A+BK_1)=\{-9.9628,\    -10.0374,\    -2.0000\}.
\end{equation*}
Регулятор найден успешно.


\subsection{Синтез ``feedforward''-компоненты}

Синтезируем «feedforward»-компоненту $K_2$ компенсирующего регулятора \eqref{eq:reg1}.
С помощью CVX решим следующую систему
\begin{equation*}
    \begin{cases}
        P\Gamma-AP=BY+B_f\\
        C_ZP=0
    \end{cases}
\end{equation*}
относительно $P$ и $Y$.