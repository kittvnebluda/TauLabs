\section{Исследование управляемости}

Рассматриваем систему:
\begin{equation*}
    \dot x=Ax+Bu,\quad
    A=\begin{bmatrix}
        -1 & 1 & 0 \\ -2 & -4 & -1 \\ 2 & 2 & -1
    \end{bmatrix},\quad
    B=\begin{bmatrix}
        2 \\ 3 \\ -1
    \end{bmatrix},\quad
    x_1=\begin{bmatrix}
        1 \\ -3 \\ 3
    \end{bmatrix}.
\end{equation*}

\subsection{Исследование управляемости}

Найдем матрицу управляемости:
\begin{equation*}
    U=[B\quad AB\quad A^2B]=
    \begin{bmatrix}
        2 & 1 & -16 \\ 3 & -15 & 47 \\ -1 & 11 & -39
    \end{bmatrix}.
\end{equation*}
Матрица полноранговая, следовательно \textbf{система управляема}.

Спектр матрицы $A$:
$$\sigma(A)=\{ -2 + j; -2-j; -2\},$$
тогда матрицы Хаутуса ($[A-\lambda_iI\quad B]$) соответственно:
\begin{equation*}
    \begin{bmatrix}
        1-j&1&0&2\\-2&-2-j&-1&3\\2&2&1-j&-1
    \end{bmatrix},\quad 
    \begin{bmatrix}
        1+j&1&0&2\\-2&-2+j&-1&3\\2&2&1+j&-1
    \end{bmatrix},\quad
    \begin{bmatrix}
        1 & 1 & 0 & 2 \\
        -2 & -2 & -1 & 3 \\
        2 & 2 & 1 & -1
    \end{bmatrix}.
\end{equation*}
Все они трехранговые, \textbf{что удовлетворяет критерию Хаутуса управляемой системы}.

Жорданова форма матрицы $A$ и $P^{-1}B$:
\begin{equation*}
    A =\begin{bmatrix}
        
-2&	   0&	   0\\
0&	  -2&	   -1\\
0&	   1&	  -2\\

    \end{bmatrix},\quad
    P^{-1}B=\begin{bmatrix}
        2.0000 \\ -0.7071 \\ -6.3640
    \end{bmatrix}.
\end{equation*}
Как видно, система в Жордановой форме полностью управляема, а значит и \textbf{исходная система
полностью управляема}.

Итого, все три критерия сошлись на том, что система полностью управляема.

\subsection{Грамиан}

Найдем Грамиан управляемости системы относительно времени $t_1=3$:
\begin{equation*}
    P(3)=\int_{0}^{3}e^{At}BB^Te^{A^Tt}dt=
    \begin{bmatrix}
        2.2838  &  0.2838  &  1.1868 \\
        0.2838   & 1.1735  & -0.7618 \\
        1.1868  & -0.7618  &  1.3500
    \end{bmatrix}.
\end{equation*}
Его спектр:
\begin{equation*}
    \sigma(P(3))=\{ 0.0034;\ 
    3.1144;\ 
    1.6894\},
\end{equation*}
все числа положительны, а значит $det(P(3))=\lambda_1\cdot\lambda_2\cdot\lambda_3>0$ и
можно найти управление, которое переводит систему



\section{Еще одно исследование управляемости}

Рассматриваем систему:
\begin{equation*}
    \dot x=Ax+Bu,\quad
    A=\begin{bmatrix}
        -1 & 1 & 0 \\ -2 & -4 & -1 \\ 2 & 2 & -1
    \end{bmatrix},\quad
    B=\begin{bmatrix}
        2 \\ -1 \\ 1
    \end{bmatrix},\quad
    x_1'=\begin{bmatrix}
        1 \\ -3 \\ 3
    \end{bmatrix},\quad
    x_1''=\begin{bmatrix}
        0\\-2\\3
    \end{bmatrix}.
\end{equation*}

\subsection{Проверка точек}


\subsection{Исследование управляемости}

Найдем матрицу управляемости:
\begin{equation*}
    U=[B\quad AB\quad A^2B]=
    \begin{bmatrix}
        2 & -3 & 2 \\ -1 & -1 & 9 \\ 1 & 1 & -9
    \end{bmatrix}.
\end{equation*}
Матрица имеет ранг 2, следовательно \textbf{система не управляема}.

Спектр матрицы $A$:
$$\sigma(A)=\{ -2 + j;\ -2-j;\ -2\},$$
тогда матрицы Хаутуса ($[A-\lambda_iI\quad B]$) соответственно:
\begin{equation*}
    \begin{bmatrix}
        1-j&1&0&2\\
        -2&-2-j&-1&-1\\
        2&2&1-j&1
    \end{bmatrix},\quad 
    \begin{bmatrix}
        1+j&1&0&2\\-2&-2+j&-1&-1\\2&2&1+j&1
    \end{bmatrix},\quad
    \begin{bmatrix}
        1 & 1 & 0 & 2 \\
        -2 & -2 & -1 & -1 \\
        2 & 2 & 1 & 1
    \end{bmatrix}.
\end{equation*}
Первые две матрицы трехранговые, что делает комплексно сопряженные корни
\textbf{управлемыми}, но а вот третья матрица имеет ранг 2, следовательно собственное
число $-2$ \textbf{не управляемо}. Значит система не управлема.

Жорданова форма матрицы $A$ и $P^{-1}B$:
\begin{equation*}
    A =\begin{bmatrix}
        
-2&	   0&	   0\\
0&	  -2&	   -1\\
0&	   1&	  -2\\

    \end{bmatrix},\quad
    P^{-1}B=\begin{bmatrix}
        0 \\ 0.7071 \\ -2.1213
    \end{bmatrix}.
\end{equation*}
Как видно, одна из клеток не управляема, а значит и \textbf{исходная система
не управляема}.

Итого, все три критерия сошлись на том, что система не управляема.

\subsection{Грамиан}

Найдем Грамиан управляемости системы относительно времени $t_1=3$:
\begin{equation*}
    P(3)=\int_{0}^{3}e^{At}BB^Te^{A^Tt}dt=
    \begin{bmatrix}
        1.1250  & -0.8750   & 0.8750\\
        -0.8750  &  0.7500 &  -0.7500\\
         0.8750   &-0.7500&    0.7500
    \end{bmatrix}.
\end{equation*}
Его спектр:
\begin{equation*}
    \sigma(P(3))=\{ 2.5640;\ 
    0.0609;\ 
    0.0000\},
\end{equation*}
все числа положительны, а значит $det(P(3))=\lambda_1\cdot\lambda_2\cdot\lambda_3>0$ и
можно найти управление, которое переводит систему
